\chapter{Resultados} \label{resultado}


Neste capítulo são apresentados, interpretados e analisados os resultados alcançados no trabalho. Será comparado as informações referentes ao custo do protótipo com as estações do mercado. Cada variável meteorológica será analisada com a exibição da sua distribuição por meio do gráfico de histograma, as estatísticas descritivas através de tabela, gráfico de box plot e aplicado o teste de hipóteses nos resultados obtidos.

\section{Comparação de Custos}

Segundo \cite{torres2015aquisiccao} uma estação meteorológica construída a partir da plataforma arduino apresenta custo de 4\% do valor de uma estação convencional de mercado. Cabe uma observação, esse trabalho em questão utilizava apenas os sensores de umidade, luminosidade e temperatura isso no ano de 2015 o custo divulgado pelo autor foi de $R\$$ 193,00. No trabalho em questão foram utilizados mais sensores e os custos podem ser observados conforme tabela \ref{tab:custos}.

\begin{table}[!h]
\centering
\begin{tabular}{|l|c|}
\hline
\textbf{Sensor}       & \textbf{Preço} \\ \hline
DHT-22                & R\$ 23,66      \\ \hline
BMP280                & R\$ 16,90      \\ \hline
Sensor DV10 e SV10    & R\$ 257,99     \\ \hline
Protoboard/Fios       & R\$ 15,90      \\ \hline
Arduino UNO           & R\$ 42,00      \\ \hline
Frete de Equipamentos & R\$ 30,00      \\ \hline
Total                 & R\$ 386,45     \\ \hline
\end{tabular}
\caption{Custo Estação Meteorológica baixo Custo }
\label{tab:custos}
\end{table}

Conforme explicações no trabalho de \cite{ocampo2019entraves} ainda não existe nenhum tipo de regulamentação para certificar as estações de superfície por parte da aeronáutica. Por isso no mercado não iremos encontrar estações com algum tipo de selo comprovando a categoria de uma determinada estação nesse cenário utilizei de estações comerciais que podem ser adquiridas via WEB para fins de comparação.

\begin{table}[!h]
\centering
\begin{tabular}{|l|c|}
\hline
\textbf{Estação Meteorológica}                               & \textbf{Preço} \\ \hline
Estação Meteorológica Vantage Vue Davis (300 metros) - K6250 & R\$ 6250,00    \\ \hline
Estação Meteorológica Nexus - 35.1075                        & R\$ 1897,50    \\ \hline
Estação de Temperatura e Umidade Davis 6382         & R\$ 3900,00   \\ \hline
Estação Meteorológica Vantage Pro2 Davis (Cabo) - K6152C     & R\$ 9000,00    \\ \hline
\end{tabular}
\caption{Fonte: site Clima e ambiente https://www.climaeambiente.com.br/}
\label{tab:custos_estacoes}
\end{table}

Na tabela \ref{tab:custos_estacoes} foi escolhido apenas estações que tenha mensurações similares a estação desenvolvida no trabalho o único sensor adicional nelas é o pluviométrico. Existem outras estações, que medem luminosidade, índices de raio UV. Entretanto, não seria muito justo comparar equipamentos que não possuem capacidades similares. É possível notar que o custo do protótipo desenvolvido nesse trabalho e de no máximo 20\% do valor de uma estação comercial.

Vale ressaltar a diferença entre protótipo, MVP(Produto mínimo viável) e produto comercial. O protótipo como é o caso desse trabalho tem por objetivo validar ou não uma hipótese e deve ser muito mais barato em custo em relação ao produto comercial. O MVP é uma etapa acima por que deve ter a hipótese comprovada e passar por validações mínimas de mercado principalmente atendendo expectativas de usuários. O produto comercial é a última escala por que possui todas as características do MVP e validações feita por órgãos de controle, no Brasil, pode citar o INMETRO(Instituto Nacional de Metrologia, Qualidade e Tecnologia) que emite um certificado de calibração dos equipamentos. A Anatel(Agência Nacional de Telecomunicações) que pode validar o sinal wireless da estação. Essa seção tem como objetivo destacar que o valor do protótipo está bem abaixo de um produto comercial.
 

\section{Análise dos dados}


\subsection{Análise da variável de temperatura}

Podemos observar no gráfico de histograma \ref{fig:dist_temperatura} sobre a variável temperatura que as medições feitas pelo protótipo tiveram poucas variações em relação a da plataforma REDEMET. 

\begin{figure} [!h]
    \centering
    \caption{Distribuição da variável Temperatura}    
    \includegraphics [scale = 0.5] {Figuras/dist_temp.png}
    \legend{Fonte: Gráfico elaborado pelo autor}
    \label{fig:dist_temperatura}
\end{figure}

É possível notar a presença de outliers nas medições feitas pelo protótipo. Facilmente observado no gráfico \ref{fig:box_temp}

\begin{figure} [!h]
    \centering
    \caption{Box Plot Temperatura}    
    \includegraphics [scale = 0.45] {Figuras/box_temp.png}
    \legend{Fonte: Gráfico elaborado pelo autor}
    \label{fig:box_temp}
\end{figure}

Além de outliers é possível ver que a mediana das temperaturas coletadas pelo protótipo foi maior que a da plataforma REDEMET e que o menor valor de temperatura foi registrado na amostra da REDEMET. É o maior valor de temperatura foi registrado na amostra do protótipo justamente o outlier. Para confirmar essas observações e possível ver as estatísticas descritivas na tabela \ref{tab:est_desc_temp_prot}.

\begin{table}[!h]
\centering
\begin{tabular}{l|c|c|}
\cline{2-3}
                                                & \multicolumn{1}{l|}{\textbf{Temperatura REDEMET}} & \textbf{Temperatura Protótipo} \\ \hline
\multicolumn{1}{|l|}{Quantidade de Observações} & 90                                                & 90                             \\ \hline
\multicolumn{1}{|l|}{Média}                     & 27.37                                             & 29.30                          \\ \hline
\multicolumn{1}{|l|}{Desvio Padrão}             & 2.27                                              & 2.33                           \\ \hline
\multicolumn{1}{|l|}{Valor Mínimo}              & 23                                                & 27                             \\ \hline
\multicolumn{1}{|l|}{Valor Máximo}              & 33                                                & 48                             \\ \hline
\end{tabular}
\caption{Dados estatísticos variável temperatura}
\label{tab:est_desc_temp_prot}
\end{table}

\setlength\parindent{2em}
Aplicando o teste de distribuições o resultado foi que as amostras não seguem o padrão da distribuição normal. Nesse caso será aplicado o cálculo estatístico de Wilcoxon, que deu como resultado que a hipótese H0 pode ser rejeitada com o pvalor de $4e-11$ um número muito próximo de zero. Nesse caso podemos concluir que para um nível de significância de 5\% as amostras das temperaturas coletas pelo protótipo e pela plataforma REDEMET não são iguais.

Ajustando os valores das observações do protótipo em -7\% com o objetivo de igualar a média da amostra com a da REDEMET e aplicando novamente o teste de Wilcoxon o resultado foi de aceitação da hipótese H0, ou seja, as amostras foram consideradas iguais. Nesse cenário podemos afirmar que é possível exister a similaridade na mensuração bastando apenas uma calibração do equipamento. 


\subsection{Análise da variável de pressão}

Na distribuição da variável de pressão podemos observar que ela foi mais uniforme e tivemos pouca diferença entre as amostras. Conforme gráfico \ref{fig:dist_pressao}

\begin{figure} [!h]
    \centering
    \caption{Distribuição da variável pressão}    
    \includegraphics [scale = 0.5] {Figuras/dist_pressao.png}
    \label{fig:dist_pressao}
\end{figure}

Pelo gráfico \ref{fig:box_plot_pressao} é possível destacar a presença de outliers nas amostras. Visualmente as amostras têm resultados bem similares. Nesse gráfico não é da para ver a mediana dos valores possivelmente foi igual algum intervalo interquartil do gráfico.

\begin{figure} [!h]
    \centering
    \caption{Box Plot Pressão}    
    \includegraphics [scale = 0.45] {Figuras/box_plot_pressao.png}
    \label{fig:box_plot_pressao}
\end{figure}

Com as estatísticas descritivas da variável de pressão merece destaque o valor máximo que foi o mesmo para às duas amostras. O valor mínimo do protótipo é um outlier tivemos médias bem similares é um desvio padrão muito baixo na amostra da plataforma REDEMET. Conforme tabela \ref{tab:est_desc_pressao_prot}

\begin{table}[!h]
\centering
\begin{tabular}{l|c|c|}
\cline{2-3}
                                                & \multicolumn{1}{l|}{\textbf{Pressão REDEMET}} & \textbf{Pressão Protótipo} \\ \hline
\multicolumn{1}{|l|}{Quantidade de Observações} & 90                                            & 90                         \\ \hline
\multicolumn{1}{|l|}{Média}                     & 1013.44                                       & 1012.43                    \\ \hline
\multicolumn{1}{|l|}{Desvio Padrão}             & 1.11                                          & 2.2                        \\ \hline
\multicolumn{1}{|l|}{Valor Mínimo}              & 1010                                          & 995                        \\ \hline
\multicolumn{1}{|l|}{Valor Máximo}              & 1015                                          & 1015                       \\ \hline
\end{tabular}
\caption{Dados estatísticos variável pressão}
\label{tab:est_desc_pressao_prot}
\end{table}

O resultado do teste de normalidade foi favorável apenas para a amostra da plataforma REDEMET enquanto os dados do protótipo não se encaixaram em uma distribuição normal. Nesse caso será utilizado o cálculo estatístico de Wilcoxon que teve como resultado a rejeição da hipótese H0 com o pvalor muito baixo $1.73e-14$. Nesse caso podemos concluir que para um nível de significância de 5\% as médias das amostras coletas para a variável pressão do protótipo e da plataforma REDEMET não são iguais.

Na variável pressão mesmo aplicando um fator de correção de +1\% nos valores das observações da amostra do protótipo com o objetivo de igualar a média da amostra e da plataforma REDEMET não foi possível obter um resultado positivo no teste de Wilcoxon. O que pode indicar que os equipamentos têm formas de mensuração diferentes e que conforme aumente a quantidade das observações a diferença entre as amostras irá aumentar.

\subsection{Análise da variável de umidade}

As amostras da distribuição da variável de umidade são bem similares com poucas variações. Conforme gráfico \ref{fig:dist_umidade}. Visualmente pelo gráfico podemos notar que as destruições não seguem características de uma distribuição normal. Nenhum percentual de umidade foi registrado abaixo de 40\%. A maior parte das medições em ambas as amostras se concentrou próximas de 80\%.

\newpage

\begin{figure} [!h]
    \centering
    \caption{Distribuição da variável Umidade}
    \includegraphics [scale = 0.5] {Figuras/dist_umidade.png}
    \label{fig:dist_umidade}
\end{figure}

Pelo gráfico \ref{fig:box_plot_umidade} podemos ver apenas um outlier na amostra do protótipo. A amostra do REDEMET tem mais variações nas medições do que a do protótipo. A mediana da amostra do protótipo é maior que a da amostra REDEMET. O maior valor de umidade foi registrado pela amostra do protótipo. O menor valor de umidade foi registrado pela plataforma REDEMET.

\begin{figure} [!h]
    \centering
    \caption{Box Plot Umidade}
    \includegraphics [scale = 0.5] {Figuras/box_plot_umidade.png}
    \label{fig:box_plot_umidade}
\end{figure}

Pela tabela \ref{tab:est_desc_umidade_prot} de dados estatísticos descritivos  podemos confirmar as observações feitas no gráfico. Podemos perceber um desvio padrão muito elevado nas amostras.

\begin{table}[!h]
\centering
\begin{tabular}{l|c|c|}
\cline{2-3}
                                                & \multicolumn{1}{l|}{\textbf{Umidade REDEMET}} & \textbf{Umidade Protótipo} \\ \hline
\multicolumn{1}{|l|}{Quantidade de Observações} & 90                                            & 90                         \\ \hline
\multicolumn{1}{|l|}{Média}                     & 73.84                                         & 77.67                      \\ \hline
\multicolumn{1}{|l|}{Desvio Padrão}             & 9.16                                          & 7.40                       \\ \hline
\multicolumn{1}{|l|}{Valor Mínimo}              & 52                                            & 55                         \\ \hline
\multicolumn{1}{|l|}{Valor Máximo}              & 89                                            & 90                         \\ \hline
\end{tabular}
\caption{Dados estatísticos variável umidade}
\label{tab:est_desc_umidade_prot}
\end{table}

Aplicando o teste de normalidade nas distribuições apenas a amostra da REDEMET pode ser classificada como normal. Aplicando o teste de Wilcoxon tivemos como resultado a rejeição da hipótese H0 com o pvalor muito baixo $2.94e-11$. Nesse caso podemos concluir que para um nível de significância de 5\% as médias das amostras coletas para a variável umidade não são iguais.

Aplicando um fator de correção de -5.5\% nos valores das observações do protótipo e aplicando novamente o teste de Wilcoxon o resultado foi de aceitação da hipótese o que indica que basta uma calibração no equipamento para que as amostras sejam consideradas iguais para um nível de significância de 5.5\%. 


\subsection{Análise da variável direção do vento}

A distribuição das amostras para essa variável foi muito similar. Foi feita uma aproximação dos dados da plataforma REDEMET para a comparação com os dados do protótipo visto que o equipamento DV10 só coleta dados nos seguintes graus (0º, 45º, 90º, 135º, 180º, 225º, 270º e 315º) que representam os pontos cardeais e colaterais. 

\begin{figure} [!h]
    \centering
    \caption{Distribuição da variável direção do vento}    
    \includegraphics [scale = 0.5] {Figuras/dist_dirvento.png}
    \label{fig:dist_dirvento}
\end{figure}

No gráfico \ref{fig:box_plot_dirvento} de box plot é possível ver muito bem a similaridade entre as amostras. Sem a presença de outlier com o valor mínimo coletado pela plataforma REDEMET. 

\begin{figure} [!h]
    \centering
    \caption{Box Plot Direção do Vento}
    \includegraphics [scale = 0.5] {Figuras/box_plot_dirvento.png}
    \label{fig:box_plot_dirvento}
\end{figure}

Na tabela de estatísticas descritivas \ref{tab:est_desc_dirvento_prot} temos médias e desvio padrão muito próximo e valores máximos iguais isso é um sinal positivo para o teste de igual de média amostrais.

\begin{table}[!h]
\centering
\begin{tabular}{l|c|c|}
\cline{2-3}
                                                & \multicolumn{1}{l|}{\textbf{Umidade REDEMET}} & \textbf{Umidade Protótipo} \\ \hline
\multicolumn{1}{|l|}{Quantidade de Observações} & 90                                            & 90                         \\ \hline
\multicolumn{1}{|l|}{Média}                     & 110.00                                        & 112.00                     \\ \hline
\multicolumn{1}{|l|}{Desvio Padrão}             & 26.34                                         & 27.90                      \\ \hline
\multicolumn{1}{|l|}{Valor Mínimo}              & 90                                            & 45                         \\ \hline
\multicolumn{1}{|l|}{Valor Máximo}              & 180                                           & 180                        \\ \hline
\end{tabular}
\caption{Dados estatísticos variável direção do vento}
\label{tab:est_desc_dirvento_prot}
\end{table}

No teste de normalidade o resultado foi que ambas as amostras não seguem o padrão da distribuição normal. Na aplicação do teste de Wilcoxon tivemos como resultado do pvalor de 0,35 bem superior ao nível de significância. Nesse caso podemos concluir que as médias das amostras do protótipo e da plataforma REDEMET são iguais para um nível de significância de 5\%.


\subsection{Análise da variável velocidade do vento}

As mensurações de velocidade vento pelo protótipo não foram bem sucedidas houve a queima do sensor que conta as rotações das pás, o reed switch, durante os experimentos desse modo conforme o gráfico \ref{fig:dist_velvento} demonstra que a amostra do protótipo foram bem pequenas. 

\begin{figure} [!h]
    \centering
    \caption{Distribuição da variável direção do vento}    \includegraphics [scale = 0.5] {Figuras/dist_velvento.png}
    \label{fig:dist_velvento}
\end{figure}


Para confirmar que as mensurações de velocidade do vento não foram suficientes pelo protótipo no gráfico \ref{fig:box_plot_velvento} notamos que a amostra foi considerada como um outlier. Enquanto a amostra da plataforma REDEMET se mostrou bem distribuída e sem a presença de outliers. Com a média entre os valores de 25\% a 50\% da amostra. É com valor máximo superior a 17,5 kt e valor mínimo próximo à média dos valores coletados pelo protótipo.

\begin{figure} [!h]
    \centering
    \caption{Box Plot Velocidade do Vento}
    \includegraphics [scale = 0.5] {Figuras/box_plot_vento.png}
    \label{fig:box_plot_velvento}
\end{figure}

Observando os valores das estatísticas descritivas na tabela \ref{tab:est_desc_velvento_prot} se confirma as observações feitas na análise do gráfico de box plot.

\begin{table}[]
\centering
\begin{tabular}{l|c|c|}
\cline{2-3}
                                                & \multicolumn{1}{l|}{\textbf{Umidade REDEMET}} & \textbf{Umidade Protótipo} \\ \hline
\multicolumn{1}{|l|}{Quantidade de Observações} & 90                                            & 90                         \\ \hline
\multicolumn{1}{|l|}{Média}                     & 8.94                                          & 0.011                      \\ \hline
\multicolumn{1}{|l|}{Desvio Padrão}             & 3.56                                          & 0.01                       \\ \hline
\multicolumn{1}{|l|}{Valor Mínimo}              & 1                                             & 0                          \\ \hline
\multicolumn{1}{|l|}{Valor Máximo}              & 18                                            & 1                          \\ \hline
\end{tabular}
\caption{Dados estatísticos variável velocidade do vento}
\label{tab:est_desc_velvento_prot}
\end{table}

É evidente que a distribuição do protótipo não é normal aplicando o teste de normalidade essa observação se confirmou e a distribuição da REDEMET foi classificada como normal. Foi aplicado o teste de Wilcoxon mesmo já sabendo que o resultado seria rejeição que se confirmou com um pvalor de 0,0014. É possível concluir que as médias das amostras do protótipo e da plataforma REDEMET não são iguais para um nível de significância de 5\%.

O outliers da amostra do protótipo ocorreram devido em um instante de tempo de aproximadamente 30 minutos que a plataforma Thingspeak ficou sem receber os dados coletados quando houve o retorno da plataforma as mensurações foram enviadas de forma equivocada a pressão no lugar da temperatura. A pressão é umidade também ficaram com valores equivocados. Conforme figura do dataset \ref{fig:outlier}

\begin{figure} [!h]
    \centering
    \caption{Outlier da Amostra do Protótipo}    
    \includegraphics [scale = 0.5] {Figuras/outlier.png}
    \label{fig:outlier}
\end{figure}


