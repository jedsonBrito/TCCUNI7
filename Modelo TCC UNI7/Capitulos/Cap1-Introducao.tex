\chapter{Introdução} \label{Introducao}

Segundo a Instrução do Comando da Aeronáutica \cite{ICA:100-37} o farol rotativo de aeródromo, quando existente, deverá permanecer ligado entre o pôr e
o nascer do sol nos aeródromos com operação contínua (H24), ou seja, o foral rotativo além de usar luzes para comunicar diversas situações para tripulantes e profissionais do aeroporto ele também demarca o início do período noturno. 

As operações a noite em um aeroporto exigem o acionamento da sinalização luminosa especificos para o período. Segundo \cite{RBAC:154} a sinalização luminosa pode ser definida como a informação aeronáutica que compõe os auxílios visuais
à navegação aérea composta por todas as luzes de pista de pouso e decolagem, de pista de táxi e de pátio de aeronaves.

As luzes são um importante conjunto de auxílios visuais no aeroporto. Elas podem demarcar o limite da área de pouso. Diferentes cores são usadas para indicar o início da pista, o meio, a proximidade com o fim e o local exato onde termina a pista. Uma forma de detectar o funcionamento de todas as luzes é feita pela inspeção de pista que um procedimento normalmente descrito no Manual de Operações aeroportuárias(MOPS) que é especifico de cada aeroporto. 

A inspeção de pista é um procedimento complexo que envolte verificação de diversos componentes principalmente na área de movimento do aeroporto e pode ser realizada várias vezes durante o dia. Esse trabalho tem como objetivo propor a criação de um sistema para auxiliar a verificação de luzes nas laterais das pistas principais e taxiways com a finalidade de auxilar o procedimento de inspeção de pista.


\section{Problema}

O funcionamento correto das luzes em pistas e taxiways e de fundamental importância para a segurança operacional do aeroporto. Esse trabalho pretende criar um mecanismo para auxiliar a verificação da iluminação correta nas laterais da pistas e taxiways de aeroportos. O trabalho irá pesquisa a existência de sistemas que utilizem sensores para a detecção do correto grau de iluminação e também para verificar se determinada luz está ou não queimada e aplica-lo ao contexto aeroportuário na ausência de uma sistema adequado o trabalho irá propor a criação de um sistema. 

Existe diversas complexidades que esse sistema deverá sobrepor como os marcos de regulamentação de aviação civil brasileira, as condições meterológicas adversas que o sistema deve suportar e também ser um sistema de fácil manutenção e baixo custo.

O trabalho e bem importante pois o não funcionamento correto das luzes podem acarretar no fechamente temporário do aerodrómo até o que problema seja resolvido. Um dos meios para evitar os problemas seria o uso de mecanimos de detecção de forma preventiva. 

\section{Hipóteses}

O trabalho vai investigar se o uso de sensores mais comuns do mercado da plataforma Arduino que podem ser uma ferramenta auxiliar para a detecção de funcionamento de luzes de balizamento laterais de runways e taxiways.

\section{Objetivos}

O objetivo geral desse trabalho é construir um sistema para detectar o funcionamento correto de luzes de balizamente de taxiways e runways em aeroportos. Os objetivos especificos desse trabalho consistem em:
\begin{itemize}
    \item Analisar os principais sensores do mercado para a composição de sistema de modo a descobrir suas vantagens e limitações.
    \item Selecionar para a criação do protótipo os sensores mais adequados para a tarefa.
    \item Elaborar o protótipo do Sistema afim de comprovar sua eficiência em atigir os objetivos.
\end{itemize}

\section{Organização do Trabalho}

Este trabalho está organizado conforme a estrutura abaixo:

\begin{description}
   \item[Capítulo 1:] Introdução.
   \item[Capítulo \ref{RevisaoBibliografica}:] Revisão Bibliográfica/ Embasamento Teórico (com o referencial teórico e trabalhos relacionados).
   \item[Capítulo \ref{desenvolvimento}:] Metodologia ou Desenvolvimento (material e métodos).
   \item[Capítulo \ref{resultado}:] Resultados e Discussões.
   \item[Capítulo \ref{conclusao}:] Conclusão (e trabalhos futuros).
\end{description}


 









