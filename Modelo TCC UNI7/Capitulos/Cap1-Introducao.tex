\chapter{Introdução} \label{Introducao}


Este capítulo é a parte inicial do trabalho. Deve ser apresentado, com apoio da literatura, uma síntese do tema central do trabalho indicando o problema da pesquisa; os objetivos, a justificativa e a organização do trabalho. 


É importante ressaltar que um \textbf{trabalho de conclusão de curso} (TCC) da graduação é um ``documento que apresenta o resultado de estudo, devendo expressar conhecimento do assunto escolhido, que deve ser obrigatoriamente emanado da disciplina, módulo, estudo independente, curso, programa, e outros ministrados'' \cite{NBR14724:2011}  e  deve ser feito sob a coordenação de um orientador e/ou coorientador.

\section{Problema}

O problema é o assunto principal do trabalho. É o que se pretende abordar, o
que se quer pesquisar, a dificuldade que se procura resolver ou, em última análise, a
pergunta que a pesquisa se propõe a responder.

Ao apresentar o problema, o autor deve justificar a necessidade de pesquisálo e destacar a importância de sua solução para a ciência, a sociedade e a
instituição patrocinadora, se for o caso. É importante, também, delimitar sua
abrangência, especificando, entre outras coisas, a população, a área geográfica e o
horizonte temporal considerados no trabalho.

\section{Hipóteses}

As hipóteses são necessárias quando se trata de um trabalho em que se
procura estabelecer relações de causa e efeito entre fatos. São respostas ou
soluções presumidas (portanto, provisórias e sujeitas a teste) para o problema
apresentado. Como são “respostas”, as hipóteses devem sempre ser formuladas de
modo afirmativo.

\section{Objetivos}

Descrever o objetivo principal e os objetivos específicos da pesquisa. Os objetivos constituem a finalidade de um trabalho científico. O objetivo principal é mais amplo e deve descrever de forma clara e sucinta qual a meta que se deseja atingir (uma proposta que solucione um problema; uma proposta de melhoria; uma análise de uma situação, entre outros exemplos). 

Para cumprir o objetivo principal muitas vezes é necessário delimitá-lo em partes menores, ou seja, em objetivos específicos. Os objetivos específicos devem ser partes do trabalho, que após cumpridos pelo pesquisador, atinge-se  o objetivo principal do trabalho. Devem vir listados, apresentados por tópicos e separados por ponto-e-vírgula.

A lista a seguir sugere alguns verbos adequados para iniciar a formulação de
objetivos:

\begin{table}[h]
\centering
\begin{tabular}{|c|c|c|c|c|}
\hline
Aferir       & Comparar  & Definir      & Especificar & Mostrar     \\ \hline
Analisar     & Comprovar & Delinear     & Estabelecer & Qualificar  \\ \hline
Apontar      & Confirmar & Demonstrar   & Explicar    & Quantificar \\ \hline
Apresentar   & Conhecer  & Descrever    & Gerar       & Realizar    \\ \hline
Avaliar      & Constatar & Determinar   & Identificar & Relacionar  \\ \hline
Caracterizar & Construir & Diagnosticar & Indicar     & Verificar   \\ \hline
Classificar  & Criar     & Discutir     & Medir       & -           \\ \hline
\end{tabular}
\caption{Lista de Verbos para Objetivo}
\label{tab1:lista de verbos para objetivo}
\end{table}


\section{Organização do Trabalho}

Um parágrafo fazendo uma descrição dos capítulos restantes do documento. 

\subsection{Estrutura da Monografia}

Segue uma \textbf{sugestão} para a estrutura da monografia: 

\begin{description}
   \item[Capítulo 1:] Introdução.
   \item[Capítulo \ref{RevisaoBibliografica}:] Revisão Bibliográfica/ Embasamento Teórico (com o referencial teórico e trabalhos relacionados).
   \item[Capítulo \ref{desenvolvimento}:] Metodologia ou Desenvolvimento (material e métodos).
   \item[Capítulo \ref{resultado}:] Resultados e Discussões.
   \item[Capítulo \ref{conclusao}:] Conclusão (e trabalhos futuros).
\end{description}


 









