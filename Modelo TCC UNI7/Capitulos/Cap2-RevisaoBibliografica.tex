\chapter{Referêncial Teórico} \label{RevisaoBibliografica}

Este capítulo deve apresentar uma contextualização da sua pesquisa com um resumo das discussões já feitas por outros autores sobre o assunto abordado e os conceitos principais relativos ao tema. O nome deste capítulo  de \textbf{Revisão Bibliográfica} ou \textbf{Referêncial Teórico} deve ser acordado com seu orientador.

A revisão bibliográfica é a base que sustenta qualquer pesquisa científica e  é indispensável para a delimitação do problema em um projeto de pesquisa,  para obter uma ideia precisa sobre o estado atual dos conhecimentos sobre um tema, sobre suas lacunas e sobre a contribuição da investigação para o desenvolvimento do conhecimento \cite{marconi2003}. 

Para a escrita deste capítulo, as citações e referências devem estar de acordo com a norma \cite{NBR6023:2002}, que destina-se a orientar a preparação e compilação das  referências bibliográficas de todo o documento.

\section{Trabalhos Relacionados}

Descreva os principais trabalhos realizados por outros autores sobre a temática escolhida para ser desenvolvida, apresentando os conceitos mais importantes, justificativas e características sobre o tema, do ponto de vista da análise feita pelos autores. 

É importante destacar, no contexto da pesquisa, quais os resultados já alcançados e os respectivos responsáveis e se possível uma análise  dos trabalhos consultados. Finalize a seção comparando a sua proposta de pesquisa com os trabalhos citados, destacando as semelhanças (caso existam) e a sua contribuição (o que pretende desenvolver).


\section{Fundamentação Teórica}

Criada em 07 de dezembro de 1944, por meio da assinatura da Convenção de
Chicago, a Organização de Aviação Civil Internacional (OACI) é uma agência
especializada das Nações Unidas, que tem como objetivo precípuo definir os parâmetros
mínimos aceitáveis de segurança para a aviação civil internacional. Constituída por 191
Estados Signatários da Convenção, e com sede em Montreal, Canadá, a entidade dispõe
de corpo técnico específico e instâncias consultivas nas quais atuam as autoridades de
aviação civil de seus Estados membros e diversos órgãos interessados, como associações
de classe e organizações não governamentais.
A OACI aprova normas e práticas recomendadas na aviação civil internacional
(Standards and Recommended Practices – SARPS), as quais balizam o marco regulatório
setorial dos Estados membros e a atuação de suas respectivas autoridades de aviação civil.
Há atualmente mais de 10 mil SARPs distribuídos nos 19 Anexos da Convenção de
Chicago. Por meio dessas normativas e de políticas complementares, auditorias e esforços
estratégicos de desenvolvimento, a rede global de transporte aéreo consegue operar cerca
de 100 mil voos por dia, de maneira segura e eficiente.

\nocite{Rampazzo2005,Carvalho1989}

\section{Instruções}

Este documento está em conformidade com normas vigentes da ABNT. Foi usado a classe abntex2, cuja documentação que pode ser obtida no \textit{link}  \url{http://www.abntex.net.br/} ou \url{https://www.ctan.org/pkg/abntex2} . Para melhor orientação do funcionamento da classe e ambientes, bem como possíveis alterações, recomenda-se a consulta dos manuais do abntex2 disponíveis para \textit{downloads} nos \textit{links}  em destaque. 




