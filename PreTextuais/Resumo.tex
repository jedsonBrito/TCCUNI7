%--------------------------------------------------------------------------
%--------------------- Resumo em Português --------------------------------
%--------------------------------------------------------------------------

\setlength{\absparsep}{18pt} % ajusta o espaçamento dos parágrafos do resumo
\begin{resumo}
As condições meteorológicas como vento, teto e visibilidade contribuem significativamente para os acidentes e incidentes aeronáuticos. Em razão disso, é muito importante que locais como aeroportos tenham suas próprias estações meteorológicas. No entanto, devido ao alto custo operacional desse tipo de equipamento especialmente para aeroportos regionais, poucos são os aeroportos que disponibilizam dados meteorológicos em tempo real. Diante desse cenário, esse trabalho propõe o desenvolvimento de um sistema de coleta de dados de meteorologia de baixo custo utilizando a plataforma arduino, aplicando o conceito de internet das coisas para disponibilizar os dados através da plataforma de nuvem ThingSpeak. Foi comparado dados coletados no protótipo elaborado com dados da plataforma REDEMET através de teste estatístico de comparação de amostra e verificou resultado de igualdade para a variável de direção de vento, necessidade de calibração do equipamento para temperatura e umidade, necessidade de utilizar outros sensores para pressão e velocidade do vento.


 \vspace{\onelineskip}
 \noindent
 \textbf{Palavras-chave}: 1. Estação Meteorológica. 2. Arduino. 3.ThingSpeak. 4. Sensores Meteológicos 5.IOT.

%--------------------------------------------------------------------------As palavras-chave devem estar separadas por ponto e finalizadas também por ponto. Devem ser escolhidos termos  que descrevem o conteúdo do trabalho.
\end{resumo}

%--------------------------------------------------------------------------
%--------------------- Resumo em Inglês --------------------------------
%--------------------------------------------------------------------------
\begin{resumo}[Abstract]
 \begin{otherlanguage*}{english}
   Meteorological conditions such as wind, roof and visibility contribute significantly to aeronautical accidents and incidents. As a result, it is very important places like airports have their own weather stations. However, due to the high operational cost of this type of equipment, especially for regional airports, few airports provide real-time meteorological information. In this context, this work proposes to develop a low-cost meteorology data collect system using the arduino platform, applying the concept of internet of things(iot) to make data available through the cloud platform ThingSpeak. It was compared data collected in the prototype elaborated with data from the REDEMET platform through statistical test of sample comparison and verified equality result for the wind direction variable, demand for calibration of temperature and humidity equipment and exchange sensors for wind speed and pressure.


   \vspace{\onelineskip}
   \noindent 
   \textbf{Keywords}: 1. Meteorological stations. 2. Arduino. 3.ThingSpeak. 4. Weather sensors 5. IOT.
 \end{otherlanguage*}
\end{resumo}